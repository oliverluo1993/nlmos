\documentclass[aps,prl,reprint,amsmath,amssymb]{revtex4-1}

\usepackage{epsfig,color,graphicx}
%\usepackage{algorithmic}
\usepackage{algorithm}
\usepackage{algpseudocode}

% Mathematical symbols
\newcommand*{\imi}{i} % imaginary i
\newcommand*{\E}{\mathrm{e}}
% DIRAC NOTATION
% bra-ket vectors
\newcommand{\ket}[1]{\ensuremath{\vert #1 \rangle}}
\newcommand{\bra}[1]{\ensuremath{\langle #1 \vert}}
\newcommand{\braket}[2]{\ensuremath{\langle #1 \vert #2 \rangle}} % bra-ket inner product
\newcommand{\ketbra}[2]{\ensuremath{\vert #1 \rangle \langle #2 \vert}} % ket-bra outer product
% operators
\newcommand{\op}[1]{\ensuremath{\hat{#1}}} % operator
\newcommand{\opsb}[2]{\ensuremath{\hat{#1}_{#2}}} % operator with subscript
\newcommand{\opsp}[2]{\ensuremath{\hat{#1}^{#2}}} % operator with superscript

% left-right arrow with text above it
\makeatletter
\newcommand\xleftrightarrow[2][]{%
  \ext@arrow 9999{\longleftrightarrowfill@}{#1}{#2}}
\newcommand\longleftrightarrowfill@{%
  \arrowfill@\leftarrow\relbar\rightarrow}
\makeatother

% inexact differential
\def\dbar{{\mathchar'26\mkern-12mu d}}

%partial derivative with some variables held constant
\newcommand{\pdc}[3]{\ensuremath{\left( \frac{\partial #1}{\partial #2} \right)_{#3}}}
\newcommand{\pddc}[3]{\ensuremath{\left( \frac{{\partial}^2 {#1}}{\partial {#2}^2} \right)_{#3}}}

%average
\newcommand{\av}[1]{\ensuremath{\left\langle{#1}\right\rangle}} % operator

%text color
\newcommand{\new}{\color{red}}
\newcommand{\blue}{\color{blue}}
\newcommand{\old}{\color{black}}

\begin{document}

\bibliographystyle{apsrev}


\title{
Direct unconstrained localization of nonorthogonal one-electron orbitals
}

\author{Ziling Luo}
\email{ziling.luo@mail.mcgill.ca}
\author{Rustam Z. Khaliullin}
\email{rustam.khaliullin@mcgill.ca}
\affiliation{Department of Chemistry, McGill University, 801 Sherbrooke St. West, Montreal, QC H3A 0B8, Canada}

\date{\today}

\begin{abstract}
Spatially localized molecular orbitals are of tremendous importance in the electronic structure theory as they are widely used to describe chemical bonding and speed up calculations.
Nonorthogonal localized molecular orbitals (NLMOs) are known to be noticeably more localized than their conventional orthogonal counterparts. 
Unfortunately, the existing methods to obtain NLMOs must pre-determine and freeze the localization center of each NLMO before its spread is minimized. 
This is done to avoid the “collapse” of the occupied subspace – a problem of NLMOs becoming linearly dependent. 
In this paper, we describe an unconstrained black-box method to localize nonorthogonal orbitals that determines the position of their centers automatically during the optimization process. 
The key to the new procedure is to construct and impose a penalty function which prevents the orbital “collapse”. 
An algorithm is proposed to adjust the strength of the penalty and produce the right balance between orthogonality and locality of NLMOs. 
The resulting method produces NLMO fast, without requiring good understanding of bonding patterns in the system (i.e. “chemical intuition”) and is demonstrated to work well a variety of molecules and materials (gamma-point only) including large systems with non-trivial bonding patterns. 
\end{abstract}

\maketitle

\section{Introduction} 

%1. Why LMOs and maximally localized Wannier functions (MLWFs) are important?

One-electron orbitals, which are defined as molecular orbitals (MOs) in the molecular systems and Bloch orbitals in the solid state field, play a paramount important role in electronic structure theory. 
Delocalized canonical molecular orbitals (CMOs) are widely implemented in current electronic structure calculations due to they have well defined energies and associated orbital energies. 

Localized MOs (LMOs) and their condensed-phase equivalent maximal localized Wannier functions (MLWFs) provide an equivalent description of the electron distribution in one-electron theories (Hartraa-Fock, Density Functional Theory) and are widely used to describe the chemical bonding patterns since the classical chemical bonding theory describes the bonding properties as the localized distributed electrons.
The conventional electronic structure methods experience high-cost and slow running time problems when applied to large systems, because of the need of the cubic scaling operation to diagonalize the Hamiltonian matrix.
To overcome the limitation, researchers have focused on developing the LS methods.
LMOs have been proposed to be an extremely promising candidate to achieve LS calculations in many local electron correlation electronic structure methods.~\cite{wang1992simple, ordejon1993unconstrained, mauri1993orbital, mauri1994electronic, stechel1994n_scaling, goedecker1994efficient}

Early developments of the LMOs were constrained to be orthogonal (OLMOs)~\cite{weinstein1971localized} which can be obtained through unitary transformations from CMOs based on extremizing the localization functions, such as Boys-Foster~\cite{boys1960construction}, Edmiston-Ruedenberg~\cite{bytautas2002electron, bytautas2003split, edmiston1963localized}, Pipek-Mezey~\cite{pipek1989a_fast}, and Von Niessen methods~\cite{niessen1972density}.
Due to the orthogonality condition of the OLMOs, the "orthogonalization tails" outside the localization center can be observed in the generated OLMOs.
These "orthogonalization tails" reduce the localization and complicated the electronic structural information transformation from one system to another.

Since nonorthogonal LMOs (NLMOs)~\cite{anderson1968self, diner1968fully, magnasco1974localized, payne1977hartree, mehler1977self} are obtained by nonsingular transformation from CMOs without the orthogonality condition, more degrees of freedom are available during the generation of NLMOs, resulting in a more localized distribution of electrons in space.
The concept of NLMOs was firs introduced by Anderson~\cite{anderson1968self} and Diner et al~\cite{diner1968fully} in 1968, but more efforts were devoted to constructing OLMOs previously.
Several efforts have been made to develop the method to construct NLMOs recently~\cite{feng2004An_efficient, liu2000nonorthogonal, peng2013effective, hoyvik2017generalising}, and researchers have proposed that NLMOs are about $10-30 \%$ more compact than OLMOs~\cite{feng2004An_efficient, liu2000nonorthogonal}.

One way to construct NLMOs is to employ the variational optimization by restricting the variational space to a local space.
The NLMO is then expanded in the variational space, which helps to eliminate the long-range tails.
However, this approach may introduce error due to the basis set truncation and lead to result in the very approximation NLMOs because the choice of the local space may be different from the true localization region.

Other studies proposed another method to obtain NLMOs by linear transformations of CMOs with relaxation of the orthogonality constraint.~\cite{feng2004An_efficient, cui2010efficient} 
The limitation of this method is that the final MOs either are still fairly orthogonal and similar localization compared with OLMOs or lead to the linear dependence between the orbitals.~\cite{feng2004An_efficient} (we will refer to the linear dependence problem as the collapse problem.)
To overcome the collapse problem, Yang and co-workers~\cite{feng2004An_efficient, cui2010efficient}  developed a method to construct NLMOs by fixing the centers of NLMOs during the minimization of the spread functional. 
The positions of the centers are predefined from the corresponding OLMOs~\cite{feng2004An_efficient} or good understanding of bonding patterns in the systems (i.e.``chemical intuition'')~\cite{cui2010efficient}.
While this method solved the linear dependence problem, it still needs more computational efforts to construct OLMOs centroids and to know the bonding properties in advance by using "chemical intuition", which may limit the application of the method.

In this paper, we propose a fully black box method to construct the NLMOs without the need of predefined centroids.
With the new approach, NLMOs can be obtained by automatic minimization the unconstrained spread functional with the  penalty functional which prevents the orbital collapse.
We successfully developed this new black-box approach to obtain NLMOs and achieved more compact and nonorthogonal MOs compared with OLMOs.

\section{Theory and algorithms}

The localization procedure starts with a set of occupied (or ZZZ virtual?) one-electron states $\ket{i_0}$. These orbitals are not assumed to be canonical or even orthogonal. The trial NLMOs are expressed in terms of these known initial states as
%
\begin{equation}
\begin{split}
\ket{j} = \ket{i_0} {A^i}_j  
\end{split}
\end{equation}
%
The conventional tensor notation is used to work with the nonorthogonal orbitals~\cite{head1998tensor}: covariant quantities are denoted with subscripts, contravariant quantities with superscripts, and summation is implied over the same orbital indices.

The objective function minimized in this work contains two terms: a conventional localization functional $\Omega_L$ and a term that penalizes unphysical states with linearly dependent occupied orbitals $\Omega_V$:
%
\begin{equation} \label{eq:fun-pen}
\begin{split}
\Omega(\mathbf{A}) = \Omega_L(\mathbf{A}) + c_V \Omega_V(\mathbf{A}), \\
\Omega_V(\mathbf{A}) = - \log \det \left[ \sigma (\mathbf{A}) \right]
\end{split}
\end{equation}
%
where $c_V > 0$ is the relative penalty strength and $\sigma$ is the NLMO overlap matrix 
%
\begin{equation}
\begin{split}
\sigma_{kl} = \braket{k}{l} = {A^j}_k \braket{j_0}{i_0}{A^i}_l = {A^j}_k \sigma_{ji}^0{A^i}_l .
%= (\mathbf{A}^\dagger \mathbf{T}^\dagger \mathbf{ S T A})_{kl},
\end{split}
\end{equation}
%
%where Greek indices denote basis set functions \ket{\mu} with overlap and T fixed coefficients of the initial one-electron states.
%
If the NLMOs are normalized the determinant of $\sigma$ varies from 1 for orthogonal states to 0 for linearly dependent states. The penalty function---the key ingredient of the proposed method---varies from 0 to $+\infty$ for these two extreme cases, making linearly dependent state inaccessible in the localization procedure with finite penalty strength $c_V$. A normalization constraint can be imposed on NLMOs if their coefficients are expressed in terms of variational parameters denoted with lowercase $\mathbf{a}$
%
\begin{equation}
\begin{split}
{A^i}_j = {a^i}_{j} ({a^k}_{j} \sigma^0_{kl}{a^l}_{j})^{-\frac{1}{2}}
\end{split}
\end{equation}

The addition of the penalty term converts the localization procedure into an unconstrained and straightforward optimization problem, in which the balance. 

We chose/adopted the following localization functional to implement our procedure 
%
\begin{equation} \label{eq:fun-loc}
\begin{split}
\Omega_L(\mathbf{A}) &= - \sum_K \sum_k \omega_K \log \vert {A^i}_k B^K_{ij} A^j_k \vert^2, \\
B^K_{ij} &= \bra{i_0} \E^{\imi \mathbf{G}_K \cdot \mathbf{\op{r}}} \ket{j_0}
\end{split}
\end{equation}
%
%RZZK - minus sign ensures minimization vs maximazation
%RZZK - use other functions?
%RZZK - Explain how Boys functions are obtained.
where . This functional is used for both periodic and gas-phase systems~\cite{berghold2000general}. In the latter case, it is equivalent to the Boys functional.
% berghold2000general --- PHYSICAL REVIEW B VOLUME 61, NUMBER 15

There exists multiple algorithms to carry out the optimization. In this work, we utilize a simple conjugate gradient algorithm summarized in Figure~\ref{fig:cg}. 

The algorithm requires the gradient of $\Omega$ with respect to variational parameters $\mathbf{a}$.
%
\begin{equation} \label{eq:grad}
\begin{split}
{G_i}^j \equiv \frac{\partial \Omega}{\partial {a^i}_j} &= fff \frac{\partial \Omega}{\partial {A^k}_l}
\end{split}
\end{equation}
%

%
\begin{equation} \label{eq:grad}
\begin{split}
\frac{\partial \Omega_L}{\partial {A^k}_l} &= 
\end{split}
\end{equation}
%

%
\begin{equation} \label{eq:grad}
\begin{split}
\frac{\partial \Omega_V}{\partial {A^k}_l} &= 
\end{split}
\end{equation}
%

How to choose the coefficient? $Grad = A + c B$. Choose c that minimizes the squared norm of the Grad to get
%
\begin{equation} \label{eq:grad}
\begin{split}
c = - \frac{Trace (A^{\dagger} B)}{Trace(B^{\dagger} B)}  
\end{split}
\end{equation}
%

\begin{figure}
\begin{algorithm}[H]
  \caption{Conjugate gradient minimization of $\Omega$}
  \label{alg:cg}
   \begin{algorithmic}[1]
   	%\Procedure{ALMO.Stage2}{$\mathbf{r}, \tau, \mathbf{T}_0$}
	\State Input $\epsilon_{\text{CG}}$ \Comment{Localization convergence threshold}
	%\State Input $N_{\text{CG}}$, $N_{\text{Outer}}$ \Comment{Max CG, outer iterations}
	\State Input $\epsilon_\text{Det}$ \Comment{Minimum allowed NLMO determinant}
	%\State $\mathbf{\sigma,R} \gets \text{Eq.\ref{eq:dm}}(\mathbf{T}_{0})$
	%\State $\mathbf{K}^{R_c} \gets ([\mathbf{S}]^{R_c})^{-1}$ \Comment{Inverted blocked AO overlap}
   	\State Input $\mathbf{T}_0$ \Comment{Coefficients of the initial states $\ket{i_0}$}
   	\State Input $\mathbf{S}$ \Comment{Basis set overlap}
   	\State Input $\mathbf{L}^K$ \Comment{Basis set representation of the localization operator} 
   	%RZZK: Localization matrix L is not defined. Loop over K?
   	\State $\mathbf{\sigma}_0 \gets \mathbf{T}_0^{\dagger} \mathbf{ST}_0$ \Comment{Initial orbital overlap} 
   	\State $\mathbf{B}^{K} \gets \mathbf{T}_0^{\dagger} \mathbf{L}^K \mathbf{T}_0$ \Comment{Initial localization matrix} 
	\State $\mathbf{a} \gets \mathbf{I}$ \Comment{Guess variational DOFs}
	%\State Converged $\gets$ False
	\State StopOuter $\gets$ False \Comment{Flag to exit the outer loop}
	\State $i_{\text{Outer}} \gets 0$ \Comment{Iteration counter}
	\Repeat \Comment{Loop to change the penalty strength}
		\State $i_{\text{Outer}} \gets i_{\text{Outer}} + 1$ 
		\If{$i_{\text{Outer}}>1$} \Comment{$c_V$ is initialized below}
			\State $c_{V} \gets c_V / 2$ 
		\EndIf
		\State StopCG $\gets$ False \Comment{Flag to exit the CG loop}
		\State $i_{\text{CG}} \gets 0$ \Comment{Iteration counter}
%		\State $\beta \gets 0$ \Comment{Reset conjugation}
		\Repeat \Comment{Fixed-penalty localization loop}
			\State $i_{\text{CG}} \gets i_{\text{CG}} + 1$ 
			\State $\mathbf{A} \gets \mathbf{a} \left[ \text{diagonal}(\mathbf{a}^{\dagger} \mathbf{\sigma}_0 \mathbf{a}) \right]^{-\frac{1}{2}}$ \Comment{Update NLMOs}
			\State $\mathbf{\sigma} \gets \mathbf{A}^{\dagger}\mathbf{\sigma}_0 \mathbf{A}$ \Comment{Update overlap}
			\State $\text{Det} \gets \text{determinant} (\mathbf{\sigma})$ \Comment{Determinant}
			\State $\Omega_{V} \gets - \log [\text{Det}] $ \Comment{Orthogonalization functional}
			\State $\mathbf{G}_{V} \gets \text{Eq.\ref{eq:grad-pen}}(\mathbf{A})$ \Comment{Orthogonalization gradient}
			\State $\Omega_{L} \gets \text{Eq.\ref{eq:fun-loc}}(\mathbf{A})$ \Comment{Localization functional}
			\State $\mathbf{G}_{L} \gets \text{Eq.\ref{eq:grad-loc}}(\mathbf{A})$ \Comment{Localization gradient}
			\If{$i_{\text{Outer}}=1$} %RZZK: how to compute the coefficient correctly using L2 norm
				%\State $c_{V} \gets \vert\vert \mathbf{G}_{L} \vert \vert_{2} / \vert\vert \mathbf{G}_{V} \vert \vert_{2} $ 
				\State $c_{V} \gets \text{Tr}(\mathbf{G}_L^{\dagger} \mathbf{G}_V)/\text{Tr}(\mathbf{G}_V^{\dagger}\mathbf{G}_V)$
			\EndIf
			\State $\Omega \gets \Omega_{L} + c_V \Omega_{V} $ 
			\If{$i_{\text{CG}}>1$}
				\State $\mathbf{\Gamma} \gets \mathbf{G}$ \Comment{Save old gradient}
			\EndIf 
			\State $\mathbf{G} \gets \mathbf{G}_{L} + c_V \mathbf{G}_{V} $ 
			%\State $\text{Error}_\text{CG} \gets \vert\vert \mathbf{G} \vert \vert_{\text{max}}$
			\If{$\vert\vert \mathbf{G} \vert \vert_{\text{max}} < \epsilon_{\text{CG}}$}
				\State StopCG $\gets$ True
			\EndIf
%			\If{$i_{\text{CG}}=1$ \textbf{And} $i_{\text{Outer}}=1$}
%				\State StopCG $\gets$ False \Comment{Do first iteration}
%			\EndIf
			\If{\textbf{not} StopCG}
%				\If{$i_{\text{CG}} = 1$}
%					\State $\mathbf{P}^{R_c} \gets \text{Eq.\ref{eq:prec}}(\mathbf{F},\mathbf{M}^{R_c},\mathbf{K}^{R_c}) $\Comment{Precon.}
%				\Else
%					\State $\mathbf{O}_{R_c} \gets \mathbf{D}_{R_c}$ \Comment{Save old direction}
%				\EndIf
				\If{$i_{\text{CG}} > 1$}
					\State $\mathbf{O} \gets \mathbf{D}$ \Comment{Save old direction}
				\EndIf
%				\State $\mathbf{D}_{R_c} \gets - [(\mathbf{P}^{R_c})^{-1} \mathbf{G}^{R_c}]_{R_c}$ \Comment{Precon. grad.}
				\State $\mathbf{D} \gets - \mathbf{G}$ \Comment{Initial direction}
				\If{$i_{\text{CG}}>1$}
					\State $\beta \gets \text{Tr}(\mathbf{G}^{\dagger} \mathbf{D})/\text{Tr}(\mathbf{\Gamma}^{\dagger}\mathbf{O})$
					\State $\mathbf{D} \gets \mathbf{D} + \beta \mathbf{O}$ \Comment{Search direction}
				\EndIf 
				\State $\alpha \gets \text{LineSearch}(\mathbf{D})$ \Comment{Step size}
				\State $\mathbf{a}\gets \mathbf{a} + \alpha \mathbf{D}$ \Comment{Update variational DOFs}
			\EndIf
		\Until{StopCG} 
		%\If{$\text{Det} < \text{Det}_{\text{Target}}$ \textbf{or} $i_{\text{Outer}} > N_{\text{Outer}}$}
		\If{$\text{Det} < \epsilon_{\text{Det}}$}
			\State StopOuter $\gets$ True
		\EndIf
	\Until{StopOuter}
	\State $\mathbf{return}$ $\mathbf{T} \gets \mathbf{T}_0 \mathbf{A} $ \Comment{Return NLMOs coefficients}
	%\EndProcedure
   \end{algorithmic}
\end{algorithm}
\caption{\label{fig:cg} Algorithm for the optimization of NLMOs.}
\end{figure}


\section{Results and discussion}
\begin{table}[ht]
\caption{The Localization functional, final determinant and relative percentage of NLMOs compared with OLMOs}
\centering
\begin{tabular}{c c c c c}
\hline\hline
Molecules & CMO &  OLMO$(\%)$ & NLMO$(\%)$ & Final Determinant \\
\hline
H$_2$O & 363 & 288 (20.5) & 230 (20.3) & 0.263 \\ 
CO$_2$ & 1856 & 654 (64.8) & 474 (27.5) & 0.102 \\
C$_3$H$_6$ & 2259 & 883 (60.9) & 761 (13.8) & 0.108 \\
C$_4$H$_6$ & 2951 & 1110 (62.4) & 903 (18.6) & 0.106 \\
B$_2$H$_6$ & 1727 & 676 (60.9) & 630 (6.7) & 0.718 \\
B$_3$N$_3$H$_6$ & 5471 & 1517 (72.3)  & 1242 (18.1) & 0.109  \\
C$_6$H$_6$  & 5676 & 1727 (69.6) & 1291 (25.2) & 0.107 \\ 
C$_7$H$_{16}$ & 20016 & 2220 (88.9) & 1953 (12.0) & 0.122 \\ 
C$_2$B$_{10}$H$_{12}$ & 11830 & 3432 (71.0) & 2873 (16.3) & 0.108 \\ 
C$_{20}$H$_{42}$ & 245277 & 6160 (97.5) & 5514 (10.5) & 0.108 \\ 
C$_{72}$H$_{24}$ & 351937 & 22791 (93.5) & 19201 (15.8) & 0.106 \\ 
graphene & 34998 & 8025 (77.1) & 6675 (16.9) & 0.0936 \\
Average & & 69.9 & 16.8 & \\
\hline
\end{tabular}
\label{table:nonlin}
\end{table}

\begin{figure}[htbp]
\includegraphics[scale=0.65]{figure_1.pdf} 
  \caption{The relationship between relative penalty strength with  localization functional ans determinant of C$_6$H$_6$ and C$_{72}$H$_{24}$ molecules}
\end{figure}

To test our newly generated method, we performed the simulation on several systems from simple water molecule to large polymers and graphene systems (H$_2$O, CO$_2$, C$_3$H$_6$, C$_4$H$_6$, B$_2$H$_6$, B$_3$N$_3$H$_6$, C$_6$H$_6$, C$_7$H$_{16}$, C$_2$B$_{10}$H$_{12}$, C$_{20}$H$_{42}$, C$_{72}$H$_{24}$, graphene).
We employed the DFT method to generate the NLMOs and calculate the related quantities (localization functional and determinants of CMOs, OLMOs and NLMOs).
The optimization procedure is stable and relative efficient in all out tests.
The results are showed in the Table 1. 
To illustrate the extent of the localization,  the Boys localization functional is calculated to compare between CMOs, OLOMOs and NLMOs.
The results of localization functional of the CMOs, OLMOs and NLMOs are tabulated in columns of 2-4 of table 1, respectively.
The relative decreases of the localization functions of the NLMOs to OLMOs are also reported in the table, which are from $6.7\%$ to $27.5\%$ and $16.8\%$ on average.
The determinant is reported to describe the orthogonality, which is related to the linear dependence issue.
The final determinant value has showed that all obtained NLMOs are linear independent and do not collapse onto each other.
The relationships between the relative penalty strength with determinant and localization functional have been reported in the figure 2. Even though the final determinant could go down to extremely small value (such as $10^{-7}$ ) in some systems, we set 0.1 as final determinant cutoff since we suspect some generated NLMOs may become linear dependent.
 
\begin{figure*}[hbpt]
\centering
\includegraphics[width=\textwidth]{figure_2.pdf}
\caption{The localization function vs Volume penalty coefficient of tested molecules}
\end{figure*}

\begin{figure}[htbp]
\includegraphics[scale=0.5]{figure_3.pdf} 
  \caption{The relationship between relative penalty strength with  localization functional ans determinant of C$_6$H$_6$ and C$_{72}$H$_{24}$ molecules}
\end{figure}

By visualizing the generated NLMOs and OLMOS, we compare the spatial difference between them with the classical chemical bonding principles.
We present several representative examples and show the shape of NLMOs and OLMOs.
After comparing the OLMOs and NLMOs, we found the spacial distribution of OLMOs and NLMOs are similar, which both exhibit traditional chemical bonding properties.
The shape of the correpsonding OLMO and NLMO are extremely similar  but with more compact distribution because of smaller "orthogonalization tails" for most of the tested systems except for graphene. 
In the graphene system, the generated NLMOs are more localized and reasonable representative of the $\pi$ bond than the OLMOs due to its more localized property.

Here we chose one NOLMO to illustrate this property, for example, the 100th NOLMO of “MOL” plotted in Figure 6 with a 0.1 isosurface value and with a 0.0001 isosurface value. This NOLMO is localized in an extremely small portion of real space. A similar characteristic is also observed for other NOLMOs of the studied systems. This advantage can be used to reduce the computational effort of electronic structure calculations based on real-space grids.


Present localization paths for two systems: simple and challenging. Use $c_V$ values determined by our automatic procedure. Remember that the whole point of creating this method was to make the localization black-box.

Can we localize virtual orbitals?

Present a table that compares orthogonal and nonorthogonal results, final determinants.

Show localization orbitals for a big molecule, molecule with non-intuitive electronic structure, and a periodic material.

Discuss existing issues: convergence rate, sigma-pi mixing, orthogonalization tails, sparsity of the final coefficients. Mention possible future work to resolve them.

TODO: Visualize an xyz file with localization centers to make sure NLMOs are physical. Compute the $c_V$ coefficient correctly the equation to minimize the L2 norm of the gradient.

check the coordinated of the localized orbital centres???

\section{Conclusions}


\section{Computational details}


\section{Acknowledgments} 

The research was funded by the Natural Sciences and Engineering Research Council of Canada (NSERC) through Discovery
Grants (RGPIN-2016-0505). The authors are grateful to Compute Canada and, in particular, the McGill HPC Centre for computer time.

\bibliographystyle{apsrev4-1}
\bibliography{NLMO}

\end{document}
