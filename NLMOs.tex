\documentclass[aps,prl,reprint,amsmath,amssymb]{revtex4-1}

\usepackage{epsfig,color,graphicx}
%\usepackage{algorithmic}
\usepackage{algorithm}
\usepackage{algpseudocode}

% Mathematical symbols
\newcommand*{\imi}{i} % imaginary i
\newcommand*{\E}{\mathrm{e}}
% DIRAC NOTATION
% bra-ket vectors
\newcommand{\ket}[1]{\ensuremath{\vert #1 \rangle}}
\newcommand{\bra}[1]{\ensuremath{\langle #1 \vert}}
\newcommand{\braket}[2]{\ensuremath{\langle #1 \vert #2 \rangle}} % bra-ket inner product
\newcommand{\ketbra}[2]{\ensuremath{\vert #1 \rangle \langle #2 \vert}} % ket-bra outer product
% operators
\newcommand{\op}[1]{\ensuremath{\hat{#1}}} % operator
\newcommand{\opsb}[2]{\ensuremath{\hat{#1}_{#2}}} % operator with subscript
\newcommand{\opsp}[2]{\ensuremath{\hat{#1}^{#2}}} % operator with superscript

% left-right arrow with text above it
\makeatletter
\newcommand\xleftrightarrow[2][]{%
  \ext@arrow 9999{\longleftrightarrowfill@}{#1}{#2}}
\newcommand\longleftrightarrowfill@{%
  \arrowfill@\leftarrow\relbar\rightarrow}
\makeatother

% inexact differential
\def\dbar{{\mathchar'26\mkern-12mu d}}

%partial derivative with some variables held constant
\newcommand{\pdc}[3]{\ensuremath{\left( \frac{\partial #1}{\partial #2} \right)_{#3}}}
\newcommand{\pddc}[3]{\ensuremath{\left( \frac{{\partial}^2 {#1}}{\partial {#2}^2} \right)_{#3}}}

%average
\newcommand{\av}[1]{\ensuremath{\left\langle{#1}\right\rangle}} % operator

%text color
\newcommand{\new}{\color{red}}
\newcommand{\blue}{\color{blue}}
\newcommand{\old}{\color{black}}

\begin{document}

\bibliographystyle{apsrev}


\title{
Direct unconstrained localization of nonorthogonal one-electron orbitals
}

\author{Ziling Luo}
\email{ziling.luo@mail.mcgill.ca}
\author{Rustam Z. Khaliullin}
\email{rustam.khaliullin@mcgill.ca}
\affiliation{Department of Chemistry, McGill University, 801 Sherbrooke St. West, Montreal, QC H3A 0B8, Canada}

\date{\today}

\begin{abstract}
Spatially localized one-electron orbitals are of tremendous importance in electronic structure theory of materials and molecular systems. 
They are used to interpret chemical bonding and speed up computations. 
\end{abstract}

\maketitle

\section{Introduction} 

Why LMOs and maximally localized Wannier functions (MLWFs) are important?

Why NLMOs and nonorthogonal Wannier functions are used? What are their advantages compared to their orthogonal counterparts?

What methods exists for constructing NLMOs? 

Why do we introduce a new method? What are do we hope to achieve?

Brief idea of our approach: unconstrained fully automatic localization with a penalty that prevents orbital collapse.

Take a look at this paper. 
Do they do anything similar to our method? 
``Anikin and co-workers~\cite{anikin2004} used a penalty function to enforce non-orthogonality while solving for the system variationally.''
% N. A. Anikin, V. M. Anisimov, V. L. Bugaenko and A. M. Andreyev, J. Chem. Phys., 2004, 121, 1266.


\section{Theory and algorithm}

In addition to the conventional localization function
%
\begin{equation}
\begin{split}
\Omega_L (\mathbf{T}) = ,
\end{split}
\end{equation}
%
the objective function proposed in this work contains a term that penalizes unphysical states with linearly dependent occupied orbitals:
%
\begin{equation}
\begin{split}
\Omega(\mathbf{T}) = \Omega_L(\mathbf{T}) - c_V \log \det (\sigma),
\end{split}
\end{equation}
%
where ... $\sigma = \mathbf{T}^\dagger \mathbf{ S T}$ is the NLMO overlap, and $c_V > 0$ is a penalty strength.

Why do we need to keep orbitals normalized?

To keep orbitals normalized we express NLMO coefficients in term of variational parameters $\mathbf{A}$
%
\begin{equation}
\begin{split}
{T^{\mu}}_{i} = A^{\mu}_{i} ({A^{\lambda}}_{i} S_{\lambda\nu}{A^{\nu}}_{i})^{-\frac{1}{2}}
\end{split}
\end{equation}

This approach converts the localization of NLMOs into an unconstrained and straightforward optimization problem. There exists multiple algorithms to carry out the optimization. In this work, we utilize a simple conjugate gradient algorithm shown in Figure~\ref{fig:cg}.


\begin{figure}
\begin{algorithm}[H]
  \caption{Conjugate gradient minimization of $\Omega$}
  \label{alg:cg}
   \begin{algorithmic}[1]
   	\State Input $t$ \Comment{Time step}
   	\State Input $N$ \Comment{Number of AIMD steps}
   	\State Input $T$ \Comment{Temperature}
   	\State Input $\gamma$ \Comment{Langevin damping coefficient}
   	\State Input $\Delta$ \Comment{Compensation for imperfect forces} \label{line:delta1}
   	\State Input $A$ \Comment{Number of atoms}
	\State $\tau\gets 0$ \Comment{Current time step}
	\State Allocate $\mathbf{r}, \mathbf{p}, \mathbf{f}$ \Comment{Matrices: positions, momenta, forces}
   	\State Input $\mathbf{r}$
	\State $\mathbf{p} \gets \text{Maxwell-Boltzmann}(T)$ \Comment{Generate initial momenta}
	\State $\mathbf{f} \gets \text{ALMO.Force}(\mathbf{r},\tau)$ \Comment{Compute forces (Algorithm~\ref{alg:force})}
	\For{$\tau \gets 1$ to $N$} \Comment{Time loop}
		\For{$i \gets 1$ to $A$} \Comment{Loop over atoms}
			\State $\sigma_i \gets 2 k_B T m_i t (\gamma + \Delta)$  \label{line:delta2}
			\State $\vec{w}_{i} \gets \mathcal{N}(0,\sigma_i)$ \Comment{Vector of random numbers}
			\State $\vec{r}_i \gets \vec{r}_i + \mathrm{e}^{-\gamma t/2} \frac{t}{m_i}\vec{p}_i + \mathrm{e}^{-\gamma t/4} \frac{t}{2 m_i} \left( t \vec{f}_i + \vec{w}_i \right) $
			\State $\vec{p}_i \gets \vec{p}_i + \mathrm{e}^{\gamma t/2} \frac{t}{2} \vec{f}_i$ \Comment{Start updating momenta}
		\EndFor
		\State $\mathbf{f} \gets \text{ALMO.Force}(\mathbf{r},\tau)$ \Comment{Update forces}
		\For{$i \gets 1$ to $A$}
%			\State $\vec{p}_i \gets \mathrm{e}^{-\gamma t} \left( \vec{p}_i + \mathrm{e}^{\gamma t/2} \frac{t}{2} \vec{f}_i \right) + \mathrm{e}^{-\gamma t/2} \vec{w}_i $   \Comment{Update momenta}
			\State $\vec{p}_i \gets \vec{p}_i + \mathrm{e}^{\gamma t/2} \frac{t}{2} \vec{f}_i $
			\State $\vec{p}_i \gets \mathrm{e}^{-\gamma t} \vec{p}_i + \mathrm{e}^{-\gamma t/2} \vec{w}_i $   \Comment{Update momenta}
		\EndFor
	\EndFor
   \end{algorithmic}
\end{algorithm}
\caption{\label{fig:cg} Optimization algorithm.}
\end{figure}


\section{Results and discussion}

Present localization paths for two systems.

Present a table that compares orthogonal and nonorthogonal results, determinants.

Show localization orbitals for a big molecule, molecule with non-intuitive electronic structure, a material.


\section{Conclusions}


\section{Computational details}


\section{Acknowledgments} 

The research was funded by the Natural Sciences and Engineering Research Council of Canada (NSERC) through Discovery
Grants (RGPIN-2016-0505). The authors are grateful to Compute Canada and, in particular, the McGill HPC Centre for computer time.

\bibliography{inflation}

% N. A. Anikin, V. M. Anisimov, V. L. Bugaenko and A. M. Andreyev, J. Chem. Phys., 2004, 121, 1266.

\end{document}
