\documentclass[aps,prl,reprint,amsmath,amssymb]{revtex4-1}

\usepackage{epsfig,color,graphicx}
%\usepackage{algorithmic}
\usepackage{algorithm}
\usepackage{algpseudocode}

% Mathematical symbols
\newcommand*{\imi}{i} % imaginary i
\newcommand*{\E}{\mathrm{e}}
% DIRAC NOTATION
% bra-ket vectors
\newcommand{\ket}[1]{\ensuremath{\vert #1 \rangle}}
\newcommand{\bra}[1]{\ensuremath{\langle #1 \vert}}
\newcommand{\braket}[2]{\ensuremath{\langle #1 \vert #2 \rangle}} % bra-ket inner product
\newcommand{\ketbra}[2]{\ensuremath{\vert #1 \rangle \langle #2 \vert}} % ket-bra outer product
% operators
\newcommand{\op}[1]{\ensuremath{\hat{#1}}} % operator
\newcommand{\opsb}[2]{\ensuremath{\hat{#1}_{#2}}} % operator with subscript
\newcommand{\opsp}[2]{\ensuremath{\hat{#1}^{#2}}} % operator with superscript

% left-right arrow with text above it
\makeatletter
\newcommand\xleftrightarrow[2][]{%
  \ext@arrow 9999{\longleftrightarrowfill@}{#1}{#2}}
\newcommand\longleftrightarrowfill@{%
  \arrowfill@\leftarrow\relbar\rightarrow}
\makeatother

% inexact differential
\def\dbar{{\mathchar'26\mkern-12mu d}}

%partial derivative with some variables held constant
\newcommand{\pdc}[3]{\ensuremath{\left( \frac{\partial #1}{\partial #2} \right)_{#3}}}
\newcommand{\pddc}[3]{\ensuremath{\left( \frac{{\partial}^2 {#1}}{\partial {#2}^2} \right)_{#3}}}

%average
\newcommand{\av}[1]{\ensuremath{\left\langle{#1}\right\rangle}} % operator

%text color
\newcommand{\new}{\color{red}}
\newcommand{\blue}{\color{blue}}
\newcommand{\old}{\color{black}}

\begin{document}

\bibliographystyle{apsrev}


\title{
Direct unconstrained localization of nonorthogonal one-electron orbitals
}

\author{Ziling Luo}
\email{ziling.luo@mail.mcgill.ca}
\author{Rustam Z. Khaliullin}
\email{rustam.khaliullin@mcgill.ca}
\affiliation{Department of Chemistry, McGill University, 801 Sherbrooke St. West, Montreal, QC H3A 0B8, Canada}

\date{\today}

\begin{abstract}
Spatially localized one-electron orbitals are of tremendous importance in electronic structure theory of materials and molecular systems. 
They are used to interpret chemical bonding and speed up computations. 
\end{abstract}

\maketitle

\section{Introduction} 

Why LMOs and maximally localized Wannier functions (MLWFs) are important?

Why NLMOs and nonorthogonal Wannier functions are used? What are their advantages compared to their orthogonal counterparts?

What methods exists for constructing NLMOs? 

Why do we introduce a new method? What are do we hope to achieve?

Brief idea of our approach: unconstrained fully automatic localization with a penalty that prevents orbital collapse.

Take a look at this paper. 
Do they do anything similar to our method? 
``Anikin and co-workers~\cite{anikin2004} used a penalty function to enforce non-orthogonality while solving for the system variationally.''
% N. A. Anikin, V. M. Anisimov, V. L. Bugaenko and A. M. Andreyev, J. Chem. Phys., 2004, 121, 1266.


\section{Theory and algorithms}

The localization procedure starts with the occupied one-electron states $\ket{i_0}$ obtained in the self-consistent filed procedure. These orbitals are not assumed to be canonical or even orthogonal. The trial NLMOs are expressed in terms of these known initial states as
%
\begin{equation}
\begin{split}
\ket{j} = \ket{i_0} {A^i}_j  
\end{split}
\end{equation}
%
The conventional tensor notation is used to work with the nonorthogonal orbitals~\cite{head1998tensor}: covariant quantities are denoted with subscripts, contravariant quantities with superscripts, and summation is implied over the same orbital indices.

The objective function proposed in this work contains two terms: a conventional localization functional $\Omega_L$ and a term that penalizes unphysical states with linearly dependent occupied orbitals $\Omega_V$:
%
\begin{equation} \label{eq:fun-pen}
\begin{split}
\Omega(\mathbf{A}) = \Omega_L(\mathbf{A}) + c_V \Omega_V(\mathbf{A}), \\
\Omega_V(\mathbf{A}) = - \log \det \left[ \sigma (\mathbf{A}) \right]
\end{split}
\end{equation}
%
where $c_V > 0$ is the relative penalty strength and $\sigma$ is the overlap matrix of NLMOs 
%
\begin{equation}
\begin{split}
\sigma_{kl} = \braket{k}{l} = {A^j}_k \braket{j_0}{i_0}{A^i}_l = {A^j}_k \sigma_{ji}^0{A^i}_l .
%= (\mathbf{A}^\dagger \mathbf{T}^\dagger \mathbf{ S T A})_{kl},
\end{split}
\end{equation}
%
%where Greek indices denote basis set functions \ket{\mu} with overlap and T fixed coefficients of the initial one-electron states.
%
If the NLMOs are normalized the determinant of $\sigma$ varies from 1 for orthogonal states to 0 for linearly dependent states. The penalty function---the key ingredient of the proposed method---varies from 0 to $+\infty$, respectively, making linearly dependent state inaccessible if $c_V$ is finite. A normalization constraint can be imposed on NLMOs if their coefficients are expressed in terms of variational parameters $\mathbf{a}$
%
\begin{equation}
\begin{split}
{A^i}_j = {a^i}_{j} ({a^k}_{j} \sigma^0_{kl}{a^l}_{j})^{-\frac{1}{2}}
\end{split}
\end{equation}

This approach converts the localization of NLMOs into an unconstrained and straightforward optimization problem. 

We chose/adopted the following localization functional to implement our procedure 
%
\begin{equation} \label{eq:fun-loc}
\begin{split}
\Omega_L(\mathbf{A}) &= - \sum_K \sum_k \omega_K \log \vert {A^i}_k B^K_{ij} A^j_k \vert^2, \\
B^K_{ij} &= \bra{i_0} \E^{\imi \mathbf{G}_K \cdot \mathbf{\op{r}}} \ket{j_0}
\end{split}
\end{equation}
%
%RZZK - minus sign ensures minimization vs maximazation
%RZZK - use other functions?
%RZZK - Explain how Boys functions are obtained.
where . This functional is used for both periodic and gas-phase systems~\cite{berghold2000general}. In the latter case, it is equivalent to the Boys functional.
% berghold2000general --- PHYSICAL REVIEW B VOLUME 61, NUMBER 15

There exists multiple algorithms to carry out the optimization. In this work, we utilize a simple conjugate gradient algorithm summarized in Figure~\ref{fig:cg}. 

The algorithm requires the gradient of $\Omega$ with respect to variational parameters $\mathbf{a}$.
%
\begin{equation} \label{eq:grad}
\begin{split}
{G_i}^j \equiv \frac{\partial \Omega}{\partial {a^i}_j} &= fff \frac{\partial \Omega}{\partial {A^k}_l}
\end{split}
\end{equation}
%

%
\begin{equation} \label{eq:grad}
\begin{split}
\frac{\partial \Omega_L}{\partial {A^k}_l} &= 
\end{split}
\end{equation}
%

%
\begin{equation} \label{eq:grad}
\begin{split}
\frac{\partial \Omega_V}{\partial {A^k}_l} &= 
\end{split}
\end{equation}
%

How to choose the coefficient? $Grad = A + c B$. Choose c that minimizes the squared norm of the Grad to get
%
\begin{equation} \label{eq:grad}
\begin{split}
c = - \frac{Trace (A^{\dagger} B)}{Trace(B^{\dagger} B)}  
\end{split}
\end{equation}
%

\begin{figure}
\begin{algorithm}[H]
  \caption{Conjugate gradient minimization of $\Omega$}
  \label{alg:cg}
   \begin{algorithmic}[1]
   	%\Procedure{ALMO.Stage2}{$\mathbf{r}, \tau, \mathbf{T}_0$}
	\State Input $\epsilon_{\text{CG}}$ \Comment{Localization convergence threshold}
	%\State Input $N_{\text{CG}}$, $N_{\text{Outer}}$ \Comment{Max CG, outer iterations}
	\State Input $\epsilon_\text{Det}$ \Comment{Minimum allowed NLMO determinant}
	%\State $\mathbf{\sigma,R} \gets \text{Eq.\ref{eq:dm}}(\mathbf{T}_{0})$
	%\State $\mathbf{K}^{R_c} \gets ([\mathbf{S}]^{R_c})^{-1}$ \Comment{Inverted blocked AO overlap}
   	\State Input $\mathbf{T}_0$ \Comment{Coefficients of the initial states $\ket{i_0}$}
   	\State Input $\mathbf{S}$ \Comment{Basis set overlap}
   	\State Input $\mathbf{L}^K$ \Comment{Basis set representation of the localization operator} 
   	%RZZK: Localization matrix L is not defined. Loop over K?
   	\State $\mathbf{\sigma}_0 \gets \mathbf{T}_0^{\dagger} \mathbf{ST}_0$ \Comment{Initial orbital overlap} 
   	\State $\mathbf{B}^{K} \gets \mathbf{T}_0^{\dagger} \mathbf{L}^K \mathbf{T}_0$ \Comment{Initial localization matrix} 
	\State $\mathbf{a} \gets \mathbf{I}$ \Comment{Guess variational DOFs}
	%\State Converged $\gets$ False
	\State StopOuter $\gets$ False \Comment{Flag to exit the outer loop}
	\State $i_{\text{Outer}} \gets 0$ \Comment{Iteration counter}
	\Repeat \Comment{Loop to change the penalty strength}
		\State $i_{\text{Outer}} \gets i_{\text{Outer}} + 1$ 
		\If{$i_{\text{Outer}}>1$} \Comment{$c_V$ is initialized below}
			\State $c_{V} \gets c_V / 2$ 
		\EndIf
		\State StopCG $\gets$ False \Comment{Flag to exit the CG loop}
		\State $i_{\text{CG}} \gets 0$ \Comment{Iteration counter}
%		\State $\beta \gets 0$ \Comment{Reset conjugation}
		\Repeat \Comment{Fixed-penalty localization loop}
			\State $i_{\text{CG}} \gets i_{\text{CG}} + 1$ 
			\State $\mathbf{A} \gets \mathbf{a} \left[ \text{diagonal}(\mathbf{a}^{\dagger} \mathbf{\sigma}_0 \mathbf{a}) \right]^{-\frac{1}{2}}$ \Comment{Update NLMOs}
			\State $\mathbf{\sigma} \gets \mathbf{A}^{\dagger}\mathbf{\sigma}_0 \mathbf{A}$ \Comment{Update overlap}
			\State $\text{Det} \gets \text{determinant} (\mathbf{\sigma})$ \Comment{Determinant}
			\State $\Omega_{V} \gets - \log [\text{Det}] $ \Comment{Orthogonalization functional}
			\State $\mathbf{G}_{V} \gets \text{Eq.\ref{eq:grad-pen}}(\mathbf{A})$ \Comment{Orthogonalization gradient}
			\State $\Omega_{L} \gets \text{Eq.\ref{eq:fun-loc}}(\mathbf{A})$ \Comment{Localization functional}
			\State $\mathbf{G}_{L} \gets \text{Eq.\ref{eq:grad-loc}}(\mathbf{A})$ \Comment{Localization gradient}
			\If{$i_{\text{Outer}}=1$} %RZZK: how to compute the coefficient correctly using L2 norm
				%\State $c_{V} \gets \vert\vert \mathbf{G}_{L} \vert \vert_{2} / \vert\vert \mathbf{G}_{V} \vert \vert_{2} $ 
				\State $c_{V} \gets \text{Tr}(\mathbf{G}_L^{\dagger} \mathbf{G}_V)/\text{Tr}(\mathbf{G}_V^{\dagger}\mathbf{G}_V)$
			\EndIf
			\State $\Omega \gets \Omega_{L} + c_V \Omega_{V} $ 
			\If{$i_{\text{CG}}>1$}
				\State $\mathbf{\Gamma} \gets \mathbf{G}$ \Comment{Save old gradient}
			\EndIf 
			\State $\mathbf{G} \gets \mathbf{G}_{L} + c_V \mathbf{G}_{V} $ 
			%\State $\text{Error}_\text{CG} \gets \vert\vert \mathbf{G} \vert \vert_{\text{max}}$
			\If{$\vert\vert \mathbf{G} \vert \vert_{\text{max}} < \epsilon_{\text{CG}}$}
				\State StopCG $\gets$ True
			\EndIf
%			\If{$i_{\text{CG}}=1$ \textbf{And} $i_{\text{Outer}}=1$}
%				\State StopCG $\gets$ False \Comment{Do first iteration}
%			\EndIf
			\If{\textbf{not} StopCG}
%				\If{$i_{\text{CG}} = 1$}
%					\State $\mathbf{P}^{R_c} \gets \text{Eq.\ref{eq:prec}}(\mathbf{F},\mathbf{M}^{R_c},\mathbf{K}^{R_c}) $\Comment{Precon.}
%				\Else
%					\State $\mathbf{O}_{R_c} \gets \mathbf{D}_{R_c}$ \Comment{Save old direction}
%				\EndIf
				\If{$i_{\text{CG}} > 1$}
					\State $\mathbf{O} \gets \mathbf{D}$ \Comment{Save old direction}
				\EndIf
%				\State $\mathbf{D}_{R_c} \gets - [(\mathbf{P}^{R_c})^{-1} \mathbf{G}^{R_c}]_{R_c}$ \Comment{Precon. grad.}
				\State $\mathbf{D} \gets - \mathbf{G}$ \Comment{Initial direction}
				\If{$i_{\text{CG}}>1$}
					\State $\beta \gets \text{Tr}(\mathbf{G}^{\dagger} \mathbf{D})/\text{Tr}(\mathbf{\Gamma}^{\dagger}\mathbf{O})$
					\State $\mathbf{D} \gets \mathbf{D} + \beta \mathbf{O}$ \Comment{Search direction}
				\EndIf 
				\State $\alpha \gets \text{LineSearch}(\mathbf{D})$ \Comment{Step size}
				\State $\mathbf{a}\gets \mathbf{a} + \alpha \mathbf{D}$ \Comment{Update variational DOFs}
			\EndIf
		\Until{StopCG} 
		%\If{$\text{Det} < \text{Det}_{\text{Target}}$ \textbf{or} $i_{\text{Outer}} > N_{\text{Outer}}$}
		\If{$\text{Det} < \epsilon_{\text{Det}}$}
			\State StopOuter $\gets$ True
		\EndIf
	\Until{StopOuter}
	\State $\mathbf{return}$ $\mathbf{T} \gets \mathbf{T}_0 \mathbf{A} $ \Comment{Return NLMOs coefficients}
	%\EndProcedure
   \end{algorithmic}
\end{algorithm}
\caption{\label{fig:cg} Algorithm for the optimization of NLMOs.}
\end{figure}


\section{Results and discussion}

Present localization paths for two systems: simple and challenging. Use $c_V$ values determined by our automatic procedure. Remember that the whole point of creating this method was to make the localization black-box.

Can we localize virtual orbitals?

Present a table that compares orthogonal and nonorthogonal results, final determinants.

Show localization orbitals for a big molecule, molecule with non-intuitive electronic structure, and a periodic material.

Discuss existing issues: convergence rate, sigma-pi mixing, orthogonalization tails, sparsity of the final coefficients. Mention possible future work to resolve them.

TODO: Visualize an xyz file with localization centers to make sure NLMOs are physical. Compute the $c_V$ coefficient correctly the equation to minimize the L2 norm of the gradient.

\section{Conclusions}


\section{Computational details}


\section{Acknowledgments} 

The research was funded by the Natural Sciences and Engineering Research Council of Canada (NSERC) through Discovery
Grants (RGPIN-2016-0505). The authors are grateful to Compute Canada and, in particular, the McGill HPC Centre for computer time.

\bibliography{inflation}

% N. A. Anikin, V. M. Anisimov, V. L. Bugaenko and A. M. Andreyev, J. Chem. Phys., 2004, 121, 1266.

\end{document}
