 \documentclass[aps,prl,reprint,amsmath,amssymb]{revtex4-1}

\usepackage{epsfig,color,graphicx}
%\usepackage{algorithmic}
\usepackage{algorithm}
\usepackage{algpseudocode}

% Mathematical symbols
\newcommand*{\imi}{i} % imaginary i
\newcommand*{\E}{\mathrm{e}}
% DIRAC NOTATION
% bra-ket vectors
\newcommand{\ket}[1]{\ensuremath{\vert #1 \rangle}}
\newcommand{\bra}[1]{\ensuremath{\langle #1 \vert}}
\newcommand{\braket}[2]{\ensuremath{\langle #1 \vert #2 \rangle}} % bra-ket inner product
\newcommand{\ketbra}[2]{\ensuremath{\vert #1 \rangle \langle #2 \vert}} % ket-bra outer product
% operators
\newcommand{\op}[1]{\ensuremath{\hat{#1}}} % operator
\newcommand{\opsb}[2]{\ensuremath{\hat{#1}_{#2}}} % operator with subscript
\newcommand{\opsp}[2]{\ensuremath{\hat{#1}^{#2}}} % operator with superscript

% left-right arrow with text above it
\makeatletter
\newcommand\xleftrightarrow[2][]{%
  \ext@arrow 9999{\longleftrightarrowfill@}{#1}{#2}}
\newcommand\longleftrightarrowfill@{%
  \arrowfill@\leftarrow\relbar\rightarrow}
\makeatother

% inexact differential
\def\dbar{{\mathchar'26\mkern-12mu d}}

%partial derivative with some variables held constant
\newcommand{\pdc}[3]{\ensuremath{\left( \frac{\partial #1}{\partial #2} \right)_{#3}}}
\newcommand{\pddc}[3]{\ensuremath{\left( \frac{{\partial}^2 {#1}}{\partial {#2}^2} \right)_{#3}}}

%average
\newcommand{\av}[1]{\ensuremath{\left\langle{#1}\right\rangle}} % operator

%text color
\newcommand{\new}{\color{red}}
\newcommand{\blue}{\color{blue}}
\newcommand{\old}{\color{black}}

\begin{document}

\bibliographystyle{apsrev}


\title{
Direct unconstrained localization of nonorthogonal one-electron orbitals
}

\author{Ziling Luo}
\email{ziling.luo@mail.mcgill.ca}
\author{Rustam Z. Khaliullin}
\email{rustam.khaliullin@mcgill.ca}
\affiliation{Department of Chemistry, McGill University, 801 Sherbrooke St. West, Montreal, QC H3A 0B8, Canada}

\date{\today}

\begin{abstract}
Spatially localized molecular orbitals are of tremendous importance in electronic structure theory as they are widely used to describe chemical bonding and speed up calculations.
Nonorthogonal localized molecular orbitals (NLMOs) are known to be noticeably more localized than their conventional orthogonal counterparts. 
Unfortunately, the existing methods to obtain NLMOs must pre-determine and freeze the localization center of each NLMO before its spread is minimized. 
This is done to avoid the ``collapse'' of the occupied subspace – a problem of NLMOs becoming linearly dependent. 
In this paper, we describe an unconstrained black-box method to localize nonorthogonal orbitals that determines the position of their centers automatically during the optimization process. 
The key to the new procedure is to construct and impose a penalty function which prevents the orbital ``collapse''. 
An algorithm is proposed to adjust the strength of the penalty and produce the right balance between orthogonality and locality of NLMOs. 
The resulting method produces NLMO fast, without requiring any \emph{a priori} knowledge of bonding patterns in the system (i.e. ``chemical intuition'') and is demonstrated to work well a variety of molecules and materials (RZZK: gamma-point only) including large systems with non-trivial bonding. 
\end{abstract}

\maketitle

\section{Introduction} 

% RZZK: Potential reviewers: Weitao?, Marzari, guys who developed LMO for CP2K.
%RZZK: boraphene?

Spatially localized orbitals are of paramount importance for one-electron theories such as the Hartree-Fock (HF) method and Kohn-Sham density functional theory (DFT) as well as for post-HF wavefunction-based electron correlation methods. 
Localized orbitals are widely used to describe and visualize chemical bonding between atoms thus helping classify bonds and understand electronic-structure origins of observed properties of atomistic systems. 
Furthermore, localized orbitals are the key ingredient in multiple local electronic structure methods~\cite{goedecker1999efficient, bowler2012methods, zalesny2011linear, RZK-localMP2, RZK-localCC} that dramatically reduce the computational cost of modeling electronic properties of large atomistic systems. % RZK: add citations (reviews are preferred) to local electron correlation methds like MP2 and coupled-cluster.
Spatially localized orbitals are known as localized molecular orbitals (LMOs) in the field of molecular quantum chemistry and maximally localized Wannier functions (MLWFs) in solid state physics and materials science. 
Here, they will be collectively referred to as LMOs whereas the eigenstates of the effective one-electron Hamiltonian will be called canonical molecular orbitals (CMOs) regardless of whether the system is isolated or treated with periodic boundary conditions.

%\bibitem{bowler2012methods}
%\bibinfo{author}{Bowler, D.} \& \bibinfo{author}{Miyazaki, T.}
%\newblock \bibinfo{title}{Methods in electronic structure calculations}.
%\newblock \emph{\bibinfo{journal}{Rep. Prog. Phys.}}
%  \textbf{\bibinfo{volume}{75}}, \bibinfo{pages}{036503}
%  (\bibinfo{year}{2012}).
%
%\bibitem{zalesny2011linear}
%\bibinfo{author}{Zale{\'s}ny, R.}, \bibinfo{author}{Papadopoulos, M.~G.},
%  \bibinfo{author}{Mezey, P.~G.} \& \bibinfo{author}{Leszczynski, J.}
%\newblock \emph{\bibinfo{title}{Linear-Scaling Techniques in Computational
%  Chemistry and Physics: Methods and Applications}}, vol.~\bibinfo{volume}{13}
%  (\bibinfo{publisher}{Springer Science \& Business Media},
%  \bibinfo{year}{2011}).

In traditional localization methods, LMOs are obtained by finding a unitary transformation of CMOs that extremizes a localization functional. 
%[RZZK: repetition]
Since CMOs are orthogonal and a unitary transformation preserves the overlap between the orbitals, LMOs obtained in this way are orthogonal (OLMOs) by construction~\cite{weinstein1971localized}.
Multiple localization functionals have been proposed for molecular systems including Boys-Foster~\cite{boys1960construction}, Edmiston-Ruedenberg~\cite{bytautas2002electron, bytautas2003split, edmiston1963localized}, Pipek-Mezey~\cite{pipek1989a_fast}, and Von Niessen~\cite{niessen1972density}. In condensed phase physics and materials science, RZK. 
%RZK:  menstion and cite localization methods (reviews are preferred) from condensed phase physics (there should be one review by Marzari). 
%
Due to the imposed orthogonality condition, OLMOs still exhibit small non-zero values even far away from the localization center. 
These ``orthogonalization tails'' complicate transferability of chemically relevant electronic-structure information from one system to another and, more importantly, reduce orbital locality making orbital-based local correlation methods less computationally efficient.
To mitigate the undesirable orthogonality effects, it has been proposed to lift the orthogonality constraint in the localization procedure with the goal of increasing the flexibility and locality of LMOs. 

There are two distinct approaches to construct nonorthogonal LMOs (NLMOs). One approach bypasses CMOs and directly optimizes NLMOs that are constrained to a local space in the variational self-consistent filed (SCF) procedure~\cite{RZK}. Unfortunately, the localization constraints imposed on NLMOs can result in the NLMO energy being substantially higher than that of the fully delocalized state. Another approach is to obtain NLMOs in a post-SCF localization procedure by finding a nonsingular transformation of CMOs that, unlike unitary transformation, lifts the orthogonality constraint off LMOs~\cite{anderson1968self, diner1968fully, magnasco1974localized, payne1977hartree, mehler1977self, feng2004An_efficient, cui2010efficient}.
% RZK: re-organized citations 
It has been found that NLMOs are indeed about $10-30 \%$ more localized than OLMOs~\cite{feng2004An_efficient, liu2000nonorthogonal}. 
% RZK the following (commented out) sentence is unclear: 
%The concept of NLMOs was first introduced by Anderson~\cite{anderson1968self} and Diner et al~\cite{diner1968fully} in 1968, but more efforts were devoted to constructing OLMOs previously.
Substantial recent efforts have been made to develop reliable algorithms to construct NLMOs~\cite{feng2004An_efficient, liu2000nonorthogonal, peng2013effective, hoyvik2017generalising}. % RZK: add citations for the a-priori methods?

Unfortunately, the existing methods produce NLMOs that are either still fairly similar to OLMOs (RZK: citation needed) or lead to the linear dependence between the orbitals~\cite{feng2004An_efficient}. 
The latter problem is particularly widespread and will referred here as as the collapse problem. 
To overcome this problem, Yang and co-workers~\cite{feng2004An_efficient, cui2010efficient} have developed a method to construct NLMOs by fixing the centers of NLMOs during the minimization of the spread functional. 
The positions of the centers are predefined from the corresponding OLMOs~\cite{feng2004An_efficient} or obtained good understanding of bonding patterns in the systems (i.e.``chemical intuition'')~\cite{cui2010efficient}.
While this method solves severe linear dependence issues, it still needs more computational efforts to construct OLMOs centroids and to know the bonding properties in advance, which may limit the application of the method to reltively simple systems.

%RZZK: Strictly speaking the localization of nonorthogonal orbitals is an ill-defined problem because a global search for the minimum of the localization functional will yield an electronic state 

In this paper, we propose a black-box method to construct NLMOs that determines the optimal positions of their centers automatically, without \emph{a priori} knowledge of bonding patterns in the system. 
The key new component in the proposed method is a penalty function which prevents the orbital ``collapse''. 
By adjusting the penalty strength, it is possible to find the required balance between the linear independence orthogonality and locality of the orbitals.

RZZK: The inclusion of the penalty term converts the localization procedure into an unconstrained and straightforward optimization problem. Additionally, adjusting the strength of the penalty $c_P$ enables one to achieve the right balance between the nonorthogonality and locality of the orbitals. 
An algorithm is proposed to adjust the strength of the penalty and produce the right balance between orthogonality and locality of NLMOs. 
The demonstrate that the proposed method produces NLMO fast,  (i.e. ``chemical intuition'') and is demonstrated to work well a variety of molecules and materials (RZZK: gamma-point only) including large systems with non-trivial bonding. 

\section{Theory and algorithms}

% RZK, the main question is: how closely do we follow our own recipe?

The localization procedure starts with a set of occupied  one-electron states $\ket{i_0}$. 
%RZZK: should we do virtual states as well? No, let's leave it for the next paper.
These orbitals are not assumed to be canonical or even orthogonal. 
However, they are assumed to be normalized. 
Furthermore, they must be linear independent, that is, their overlap matrix $\sigma_{ji}^0 \equiv \braket{j_0}{i_0}$ must be invertible. 
The trial NLMOs are expressed as a linear combination of these initial states
%
\begin{equation}
\begin{split}
\ket{j} = \ket{i_0} {A^i}_j  
\end{split}
\end{equation}
%
The conventional tensor notation is used to work with the nonorthogonal orbitals~\cite{head1998tensor}: covariant quantities are denoted with subscripts, contravariant quantities with superscripts, and summation is implied over the same orbital indices.

The objective function minimized in this work contains two terms: a conventional localization functional $\Omega_L$ and a term that penalizes unphysical states with linearly dependent occupied orbitals $\Omega_P$:
%
\begin{equation} \label{eq:fun-pen}
\begin{split}
\Omega(\mathbf{A}) = \Omega_L(\mathbf{A}) + c_P \Omega_P(\mathbf{A}), \\
\Omega_P(\mathbf{A}) = - \log \det \left[ \sigma (\mathbf{A}) \right]
\end{split}
\end{equation}
%
where $c_P > 0$ is the penalty strength, $\sigma$ is the NLMO overlap matrix 
%
\begin{equation}
\begin{split}
\sigma_{kl} = \braket{k}{l} = {A^j}_k \sigma_{ji}^0{A^i}_l
%= (\mathbf{A}^\dagger \mathbf{T}^\dagger \mathbf{ S T A})_{kl},
\end{split}
\end{equation}
%

If the NLMOs are normalized the determinant of $\sigma$ varies from 1 for orthogonal NLMOs to 0 for linearly dependent NLMOs. The penalty function---the key ingredient of the proposed method---varies from 0 to $+\infty$ for these two extreme cases, making linearly dependent state inaccessible in the localization procedure with finite penalty strength $c_P$. 
% RZZK: Why sigma instead of A? quadratic penalty in terms of A (more complicated in terms of a). Why log?
A normalization constraint can be imposed on NLMOs if their coefficients are expressed in terms of independent variational parameters denoted with lowercase $\mathbf{a}$
%
\begin{equation}
\begin{split}
{A^i}_j = {a^i}_{j} ({a^k}_{j} \sigma^0_{kl}{a^l}_{j})^{-\frac{1}{2}}
\end{split}
\end{equation}
%
The inclusion of the penalty term converts the localization procedure into an unconstrained and straightforward optimization problem. Additionally, adjusting the strength of the penalty $c_P$ enables one to achieve the right balance between the nonorthogonality and locality of the orbitals (see below). 

In this work, we adopted the following localization functional 
%
\begin{equation} \label{eq:fun-loc}
\begin{split}
\Omega_L(\mathbf{A}) &= - \sum_K \sum_i \omega_K \log \vert z_{i}^{K} \vert^2, \\
z_{i}^{K} &= {A^m}_i B^{K}_{mn} {A^n}_i, \\
B^{K}_{mn} &= \bra{m_0} \E^{\imi \mathbf{G}_K \cdot \mathbf{\op{r}}} \ket{n_0}
\end{split}
\end{equation}
%
%RZZK - K is an upper index that is not contravariant, there is also covariant K
%RZZK - minus sign ensures minimization vs maximazation
%RZZK - use other functions?
%RZZK - Explain how Boys functions are obtained.
where $\omega_K$ , $\mathbf{G}_K$, $\mathbf{\op{r}}$ - RZZK. Summation over $K$ is written out explicitly because $K$ is not an orbital index. This functional can be used for both periodic (ZZZ at $\Gamma$-point only?) and gas-phase systems~\cite{berghold2000general}. In the latter case, it is equivalent to the Boys localization functional~\cite{RZK}.
% berghold2000general --- PHYSICAL REVIEW B, 61,

There exists a multitude of algorithms to carry out the unconstrained minimization of functional $\Omega$ with fixed $c_P$. In this work, we used a simple conjugate gradient algorithm summarized in Figure~\ref{fig:cg}. The gradient ${G_i}^j \equiv \frac{\partial \Omega}{\partial {a^i}_j}$  required in the algorithm is a sum of the localization ${L_i}^j \equiv \frac{\partial \Omega_L}{\partial {a^i}_j}$ and penalty ${P_i}^j \equiv \frac{\partial \Omega_P}{\partial {a^i}_j}$ components
%
\begin{equation} \label{eq:grad}
\begin{split}
G{_k}^{l} = L{_k}^{l} + c_P P{_k}^{l}.
\end{split}
\end{equation}
%
These components can be readily expressed in terms of the derivatives with respect to the transformation coefficients $\tilde{X}{_k}^l \equiv \frac{\partial \Omega_X}{\partial {A^k}_l}$, where $X$ is either $L$ or $P$:
%
\begin{equation} \label{eq:grad-convert}
\begin{split}
{X_i}^j & = \tilde{X}{_k}^l \frac{\partial {A^k}_l}{\partial {a^i}_j} = \tilde{X}{_i}^j \sigma_{jj}^{-\frac{1}{2}} - \sigma_{in}^0 {A^n}_j  \sigma_{jj}^{-\frac{1}{2}} {A^m}_j \tilde{X}{_m}^j
\end{split}
\end{equation}
%
\begin{equation} \label{eq:grad-loc}
\begin{split}
\tilde{L}{_k}^l & = - \sum_K \frac{4 \omega_K}{\vert z_{l}^{K} \vert^2} \times \\ 
&\times \left[  \operatorname{Re}(B^{K}_{kn}) {A^{n}}_{l} \operatorname{Re}(z_{l}^{K}) + \operatorname{Im}(B^{K}_{kn}) {A^{n}}_{l} \operatorname{Im}(z_{l}^{K}) \right]
\end{split}
\end{equation}
%
\begin{equation} \label{eq:grad-pen}
\begin{split}
\tilde{P}{_k}^l & = -2 \sigma_{km}^0 {A^m}_n \sigma^{nl} 
\end{split}
\end{equation}
%

\textbf{Penalty strength.} If the penalty strength $c_P$ is extremely large, $\Omega_L$ is negligible compared to the penalty term and the minimization of $\Omega$ is numerically equivalent to a trivial orbital orthogonalization. In the opposite case of extremely small $c_P$, the minimization of $\Omega$ may result in a linear dependence between NLMOs as reported earlier~\cite{RZK}. % RZK: cite Weitao's paper, in which they fix centroids.
In the latter case, the algorithm shown in Figure~\ref{fig:cg} fails to compute the inverse of the NLMO overlap required in Eq.~(\ref{eq:grad-pen}). 

As we show below, there is a wide range of $c_P$ values between the two extremes that produce NLMOs that are substantially more localized than OLMOs and linearly independent.  
Within this range, $c_P$ serves as an adjustable parameter that can be used to achieve a desirable locality-orthogonality compromise. 
For a fixed value of $c_P$, the degree of linear dependence between the optimal NLMOs can be measured by the value of the determinant of the NLMO overlap matrix. 
Thus, the flexibility of the penalized localization method presented here can be favorably compared to the existing methods that produce either orthogonal orbitals, linearly-dependent LMOs, or NLMOs with fixed localization centers. 

A simple strategy to find an appropriate penalty strength is to minimize $\Omega$ with a sufficiently large initial $c_P$ value and then gradually decrease $c_P$ until the determinant of the overlap of the \emph{optimal} NLMOs drops below a desired threshold $\text{D}_{\text{min}} \in (0,1]$. 
The initial value of $c_P$ should be chosen to balance approximately the localization and penalty components of $\Omega$. 
Thus, a reasonable initial value of $c_P$ can be estimated by assuming that the reduction in $\Omega_L$ upon the minimization of $\Omega$ has the same order of magnitude as the change in the penalty $c_P \Omega_P$:
%
\begin{equation} \label{eq:cp-beta}
\begin{split} 
%\Omega_L(\mathbf{I}) + c_P \Omega_P(\mathbf{I}) & \sim \Omega_{L}(\mathbf{A}^{\ast}) - c_P \log \text{D}_{\text{min}} \\
c_P^{\text{init}} & \sim \frac{ \Omega_{L}(\mathbf{I}) - \Omega_{L}(\mathbf{A}^{\ast}) }{ \Omega_{P}(\mathbf{A}^{\ast}) - \Omega_{P}(\mathbf{I}) } = \frac{ \beta_L^{\ast} }{ \beta_P^{\ast} } \times \Omega_{L}(\mathbf{I})
\end{split}
\end{equation}
%
where $\mathbf{A}^{\ast}$ denotes the (yet unknown) solution to the minimization problem, $\beta_L^{\ast} \equiv \frac{\Omega_L(\mathbf{I})- \Omega_L(\mathbf{A}^{\ast})}{\Omega_L(\mathbf{I})} \in [0,1]$ is the (positive) expected relative reduction in the localization functional, and $\beta_P^{\ast} \equiv \log \frac{\det \sigma(\mathbf{I})}{ \det \sigma(\mathbf{A}^{\ast}) } \approx \log \frac{\det \sigma(\mathbf{I})}{ \text{D}_{\text{min}} } \equiv \beta_P > 0$ is the logarithm of the ratio of the initial and final determinants. 

The importance of Eq.~(\ref{eq:cp-beta}) is that it allows to express $c_P$ as a dimensionless constant multiplied by $\Omega_L(\mathbf{I})$, which can be easily calculated in the beginning of the optimization procedure:
%
\begin{equation} \label{eq:cp-alpha}
\begin{split}
c_{P}^{\text{init}} \sim \frac{ \beta_L^{\ast} }{ \beta_P } \times \Omega_L(\mathbf{I}) \equiv \alpha \Omega_L(\mathbf{I})
\end{split}
\end{equation}
%
The equation also makes clear that the penalty component is an extensive function of a system with the units that are consistent with the localization component. Although the optimal dimensionless parameter $\beta_L^{\ast}$ is not known \emph{a priori} its magnitude can be easily estimated to obtain a sufficiently large initial guess for $c_P$. For example, an optimization of canonical orbitals $\det \sigma(\mathbf{I})=1$ that stops when the NLMO determinant drops below $\text{D}_{\text{min}} = 0.1$ can be initialized by adopting the maximum possible value of $\beta_L^{\ast} = 1$ that results in $\alpha = \log^{-1} 10$.

The procedure for tuning $c_P$ is shown as the outer loop of the algorithm in Figure~\ref{fig:cg}. It only requires $\text{D}_{\text{min}}$ as input. % and optional parameter $\alpha$ that can improve efficiency of the optimization. %RZZK

%RZZK: Strategy 2: If the precise value of the determinant is necessary the penalty can be replaced with c_P becoming a Langrange multiplier that can be determined precisely. In a MD or geopt, determine $c_P$ once and then fix it for subsequent geometries. This is valid as long as the electronic properties of the system remain the same (e.g. there is not insulator to metal phase transition). 

%RZZK: Dmin is a misnomer.

%%%{\color{red} It might seem like a good idea to choose initial $c_P$ value that minimizes the Frobenius norm of the total gradient. The motivation is that the gradient of the localization component can be balanced as much as possible by the penalty gradient. In this case, the initial $c_P$ is given by the equation below (somewhat problematic index summation). The main reason not to use this initial value is that for $P_k^l$ is zero for all orthonormal states, giving infinitely large coefficient.
%%%%
%%%\begin{equation}
%%%\begin{split}
%%%c_P = - \frac{ \sum_{nm} {L_m}^n {P_m}^n}{ \sum_{kl} {P_k}^l {P_k}^l}  
%%%\end{split}
%%%\end{equation}
%%%%
%%%}


\begin{figure}
\begin{algorithm}[H]
  \caption{Conjugate gradient minimization of $\Omega$}
  \label{alg:cg}
   \begin{algorithmic}[1]
   	%\Procedure{Minimize$\Omega$}{$c_P, \tau, \mathbf{T}_0$}
	\State Input $\epsilon_{\text{CG}}$ \Comment{Localization convergence threshold}
	%\State Input $N_{\text{CG}}$, $N_{\text{Outer}}$ \Comment{Max CG, outer iterations}
	\State Input $\text{D}_{\text{min}}$ \Comment{Minimum allowed NLMO determinant}
   	\State Input $\mathbf{T}_0$ \Comment{Coefficients of the initial states $\ket{i_0}$}
   	\State Input $\mathbf{S}$ \Comment{Basis set overlap}
   	\State Input $\mathbf{L}^K$ \Comment{Basis set representation of the localization operator} 
   	%RZZK: Localization matrix L is not defined. Loop over K?
   	\State $\mathbf{\sigma}_0 \gets \mathbf{T}_0^{\dagger} \mathbf{ST}_0$ \Comment{Initial orbital overlap} 
   	\State $\mathbf{B}^{K} \gets \mathbf{T}_0^{\dagger} \mathbf{L}^K \mathbf{T}_0$ \Comment{Initial localization matrix} 
	\State $\mathbf{a} \gets \mathbf{I}$ \Comment{Guess variational DOFs}
	%\State Converged $\gets$ False
	\State StopOuter $\gets$ False \Comment{Flag to exit the outer loop}
	\State $i_{\text{Outer}} \gets 0$ \Comment{Iteration counter}
	\Repeat \Comment{Loop to change the penalty strength}
		\State $i_{\text{Outer}} \gets i_{\text{Outer}} + 1$ 
		\State StopCG $\gets$ False \Comment{Flag to exit the CG loop}
		\State $i_{\text{CG}} \gets 0$ \Comment{Iteration counter}
%		\State $\beta \gets 0$ \Comment{Reset conjugation}
		\Repeat \Comment{Fixed-penalty localization loop}
			\State $i_{\text{CG}} \gets i_{\text{CG}} + 1$ 
			\State $\mathbf{A} \gets \mathbf{a} \left[ \text{diagonal}(\mathbf{a}^{\dagger} \mathbf{\sigma}_0 \mathbf{a}) \right]^{-\frac{1}{2}}$ \Comment{Update NLMOs}
			\State $\mathbf{\sigma} \gets \mathbf{A}^{\dagger}\mathbf{\sigma}_0 \mathbf{A}$ \Comment{Update overlap}
			\State $\text{Det} \gets \text{determinant} (\mathbf{\sigma})$ \Comment{Determinant}
			\State $\Omega_{P} \gets - \log [\text{Det}] $ \Comment{Orthogonalization functional}
			\State $\mathbf{P} \gets \text{Eq~(\ref{eq:grad-pen}) and~(\ref{eq:grad-convert})}$ \Comment{Orthogonalization gradient}
			\State $\Omega_{L} \gets \text{Eq~(\ref{eq:fun-loc})}$ \Comment{Localization functional}
			\State $\mathbf{L} \gets \text{Eq~(\ref{eq:grad-loc}) and~(\ref{eq:grad-convert})}$ \Comment{Localization gradient}
			\If{$i_{\text{Outer}}=1$ \textbf{and} $i_{\text{CG}}=1$} 
				%\State $c_{P} \gets \text{Tr}(\mathbf{L}^{\dagger} \mathbf{P})/\text{Tr}(\mathbf{P}^{\dagger}\mathbf{P})$
				\State $c_{P} \gets \Omega_{L}(\log [\text{Det} / \text{D}_{\text{min}} ])^{-1}$ \Comment{Initial strength}
			\EndIf
			\State $\Omega \gets \Omega_{L} + c_P \Omega_{P} $ 
			\If{$i_{\text{CG}}>1$}
				\State $\mathbf{\Gamma} \gets \mathbf{G}$ \Comment{Save old gradient}
			\EndIf 
			\State $\mathbf{G} \gets \mathbf{L} + c_P \mathbf{P} $ 
			%\State $\text{Error}_\text{CG} \gets \vert\vert \mathbf{G} \vert \vert_{\text{max}}$
			\If{$\vert\vert \mathbf{G} \vert \vert_{\text{max}} < \epsilon_{\text{CG}}$}
				\State StopCG $\gets$ True
			\EndIf
%			\If{$i_{\text{CG}}=1$ \textbf{And} $i_{\text{Outer}}=1$}
%				\State StopCG $\gets$ False \Comment{Do first iteration}
%			\EndIf
			\If{\textbf{not} StopCG}
%				\If{$i_{\text{CG}} = 1$}
%					\State $\mathbf{P}^{R_c} \gets \text{Eq.\ref{eq:prec}}(\mathbf{F},\mathbf{M}^{R_c},\mathbf{K}^{R_c}) $\Comment{Precon.}
%				\Else
%					\State $\mathbf{O}_{R_c} \gets \mathbf{D}_{R_c}$ \Comment{Save old direction}
%				\EndIf
				\If{$i_{\text{CG}} > 1$}
					\State $\mathbf{O} \gets \mathbf{D}$ \Comment{Save old direction}
				\EndIf
%				\State $\mathbf{D}_{R_c} \gets - [(\mathbf{P}^{R_c})^{-1} \mathbf{G}^{R_c}]_{R_c}$ \Comment{Precon. grad.}
				\State $\mathbf{D} \gets - \mathbf{G}$ \Comment{Initial direction}
				\If{$i_{\text{CG}}>1$}
					\State $\beta \gets \text{Tr}(\mathbf{G}^{\dagger} \mathbf{D})/\text{Tr}(\mathbf{\Gamma}^{\dagger}\mathbf{O})$
					\State $\mathbf{D} \gets \mathbf{D} + \beta \mathbf{O}$ \Comment{Search direction}
				\EndIf 
				\State $\alpha \gets \text{argmin}_{\alpha} \Omega(\mathbf{a} + \alpha \mathbf{D})$ \Comment{Line search}
				\State $\mathbf{a}\gets \mathbf{a} + \alpha \mathbf{D}$ \Comment{Update variational DOFs}
			\EndIf
		\Until{StopCG} 
%RZK: add an additional criterion that prevents high-determinant cases go on forever
		%\If{$\text{Det} < \text{Det}_{\text{Target}}$ \textbf{or} $i_{\text{Outer}} > N_{\text{Outer}}$}
		\If{$\text{Det} < \text{D}_{\text{min}}$}
			\State StopOuter $\gets$ True
		\EndIf
		\If{$i_{\text{Outer}}>1$} 
			\State $c_{P} \gets c_P / 2$ \Comment{Reduce $c_P$}
		\EndIf
	\Until{StopOuter}
	\State $\mathbf{return}$ $\mathbf{T} \gets \mathbf{T}_0 \mathbf{A} $ \Comment{Return NLMOs coefficients}
	%\EndProcedure
   \end{algorithmic}
\end{algorithm}
\caption{\label{fig:cg} Algorithm for the optimization of NLMOs.}
\end{figure}

\section{Implementation and computational details}

All calculations were preformed using DFT as implemented in the CP2K software package~\cite{cp2kgeneral}.
The energy of all molecules was evaluated using periodic boundary conditions.
CP2K relies on the mixed Gaussian and plane-wave representation of the electronic degree of freedom~\cite{hutter2014cp2k}.
The localized atom-centered Gaussian basis sets were used for constructing molecular orbitals and the plane waves were used to construct the Kohn-Sham matrix efficiently (RZK: KS is irrelevant).
The Becke-Lee-Yang-Parr generalized gradient approximation~\cite{becke1988density, lee1988development} was used as the exchange-correlation functional.
Goedecker-Teter-Hutter pseudopotentials~\cite{goedecker1996separable} were used together with a triple-$\zeta$ Gaussian basis set with two sets of polarization functions for all atoms.
The energy cutoff of 200 Ry (RZK: seems low, check) was used to define the plane-wave basis set.
The integration over the Brillouin zone was performed using the gamma point.
We chose the final determinant cutoff as 0.09~a.u., so we dismiss all generated NLMOs with final determinant less than 0.09 to prevent orbitals from collapsing.



\section{Results and discussion}

% RZK: report percentages obtained with the previous method. Do we need absolute values of the localization functioal?
\begin{table}[ht]
\caption{The Localization functional, final determinant and relative percentage of NLMOs compared with OLMOs}
\label{tab:loc}
\centering
\begin{tabular}{l c c c c}
\hline\hline
Molecules & CMO &  OLMO$(\%)$ & NLMO$(\%)$ & $\det(\sigma)$ \\
\hline
%RZZK: use beta values defined in the equations above, a priori and posteriori values of alpha
H$_2$O & 363 & 288 (20.5) & 230 (20.3) & 0.263 \\ 
CO$_2$ & 1856 & 654 (64.8) & 474 (27.5) & 0.102 \\
B$_2$H$_6$ & 1727 & 676 (60.9) & 630 (6.7) & 0.718 \\
B$_3$N$_3$H$_6$ & 5471 & 1517 (72.3)  & 1242 (18.1) & 0.109  \\
C$_2$B$_{10}$H$_{12}$ & 11830 & 3432 (71.0) & 2873 (16.3) & 0.108 \\ 
C$_3$H$_6$ & 2259 & 883 (60.9) & 761 (13.8) & 0.108 \\
C$_4$H$_6$ & 2951 & 1110 (62.4) & 903 (18.6) & 0.106 \\
C$_6$H$_6$  & 5676 & 1727 (69.6) & 1291 (25.2) & 0.107 \\ 
C$_7$H$_{16}$ & 20016 & 2220 (88.9) & 1953 (12.0) & 0.122 \\ 
C$_{20}$H$_{42}$ & 245277 & 6160 (97.5) & 5514 (10.5) & 0.108 \\ 
C$_{72}$H$_{24}$ & 351937 & 22791 (93.5) & 19201 (15.8) & 0.106 \\ 
graphene & 34998 & 8025 (77.1) & 6675 (16.9) & 0.0936 \\
Average & & 69.9 & 16.8 & \\
\hline
\end{tabular}
\label{table:nonlin}
\end{table}

% RZK: The graphs are of poor quality, line styles must be changed. What points are we going to include? They must be in agreement with our $c_P$ adjustment algorithm.
% RZK: We do not need Max determinant because the position of the line is obvious.
% RZK: We need to reference the algortihm used for OLMO localization because the OLMO line is higher than the maximum in our line.
\begin{figure}[htbp]
\includegraphics[scale=0.6]{figure_2.pdf} 
  \caption{The relationship between relative penalty strength with localization functional and determinant of C$_6$H$_6$ and C$_{72}$H$_{24}$ molecules}. RZK: Which one is which?
\end{figure}

This paragraph should be in the Computational details, not results and discussion. To test our newly generated method, we performed the simulations on several systems from simple water molecule to large polymer and graphene systems (H$_2$O, CO$_2$, C$_3$H$_6$, C$_4$H$_6$, B$_2$H$_6$, B$_3$N$_3$H$_6$, C$_6$H$_6$, C$_7$H$_{16}$, C$_2$B$_{10}$H$_{12}$, C$_{20}$H$_{42}$, C$_{72}$H$_{24}$, graphene). %RZK: Unless a system is trivial, it must be properly named and described in sufficient detail so the reader know what molecule you are talking about.

%RZK: What is our motivation in choosing these systems? For simple systems, compare the performance of our method to the previously reported localization method of ZZZ. The main question is whether relaxation of the centroid positions allows to achieve better localization. We would like to see whether the new scheme works for more complex system, in which it is difficult to predict \emph{a priori} the position of the centroids.

%This is already discussed in Computational details: We employed the DFT method to generate the NLMOs and calculate the related quantities (localization functional and determinants of CMOs, OLMOs and NLMOs).
The localization procedure is stable and relatively efficient in most of our tests.
The results are shown in the Table~\ref{tab:loc}.
To illustrate the extent of the localization, the Boys localization functional is calculated to compare between CMOs, OLOMOs and NLMOs. [RZK: We are not doing the Boys localization.]

Yang et al.~\cite{feng2004An_efficient, cui2010efficient} have previously shown that, based on their prefixed centers algorithms, the NLMOs are more compact than the corresponding OLMOs with about $10\%$ to $30\%$ in the value of spread functional.
Our algorithms gave the relatively similar results with theirs.
%RZK: do not use table/figure numbers directly. Use \ref, as I demostrated above. 
The results of localization functional of the CMOs, OLMOs and NLMOs are tabulated in columns of 2-4 of Table 1, respectively.
The relative decreases of the localization functions of the OLMOs to CMOs and NLMOs to OLMOs are also reported in the table.
Among our tested systems, the relative percentage of OLMOs compared with CMOs range from $20.5\%$ to $97.5\%$, while that of NLMOs to OLMOS is from $6.7\%$ to $27.5\%$.
The average decreases of OLMOs and NLMOs localization functional are $69.9\%$ and $16.8\%$.

The determinants controlled by the penalty strength are listed in the last column of the table for all systems, which indicate the orthogonality of the obtained NLMOs.
The final determinant value has shown that all obtained NLMOs are linear independent and do not collapse into each other.
The relationships between the relative penalty strength with determinant and localization functional is illustrated in the Fig. 2.
The first vertical axis describes the change of localization functional with different penalty strength during the minimization procedure, with the second axis reports the final determinants.
The yellow dot line describes the minimized value localization functional of OLMOs, with the green dash-dot line marks as the maximum determinant 1.0, which also indicates the determinant of OLMOs.
The trend of the localization function and final determinant versus penalty strength grow in the same way as we expected.
With weakening the penalty strength, we obtained more localized and nonorthogonal MOs.
Even though the final determinant could go down to extremely small value (such as $10^{-7}$ ) in some systems, we set 0.09 as the final determinant cutoff since we suspect some generated NLMOs may become linear dependent with those extremely small determinant values.
In some cases, we do observe the scenario described in the Theory section, which results in negligible localization functional and a trivial orbital orthogonalization problem when using extremely large penalty strength at the beginning of the optimization.
According to the Fig. 2, we can conclude that our method successfully generates linear independent NLMOs based on totally black-box algorithms.

\begin{figure*}[hbpt]
\centering
\includegraphics[width=\textwidth]{figure_3.pdf}
\caption{A plot of the OLMO and NLMO of the covalent bond of carbon atoms in C$_2$B$_{10}$H$_{12}$ molecule. (isosurface = 0.05)}
\end{figure*}

Although the obtained NLMOs are substantial more localized than the NLMOs in terms of the value of localization functional, it is still not clear how much the spatial difference between them.
By visualizing the generated NLMOs and OLMOS, we can show how much the different those orbitals look like.
Figure 3 and 4 are a few NLMOs and OLMOs obtained from our tests.
Figure 3 contains the OLMO and NLMO of carbon atoms covalent bond for C$_2$B$_{10}$H$_{12}$ molecule with 0.05 isosurface value.
The shape of the corresponding OLMO and NLMO are extremely similar, but the OLMO occupied more space compared with the NLMO, which also indicated the obtained NLMO is more compact than the OLMO.
The Previous study has proved that two adjacent C atoms form 3 center 2 electron B–C–B bonds and a classical C–C bond in ortho-carborane molecule.\cite{melichar2018systematic}
It can be seen from the Fig.3 that both OLMO and NLMO can well present the complicated bonding pattern, without requiring "chemical intuition" and pre-determined orbital centers in the system.
The obtained NLMOs and OLMOs  are completely equivalent to each other and to the CMOs in their representation of the electronic structure.

%RZK: remove grpahene image. Draw the periodic cell on one of the images. Same for the Zhenzhe's polymer.
\begin{figure*}[hbpt]
\centering
\includegraphics[width=\textwidth]{figure_4.pdf}
\caption{A plot of the OLMO and NLMO of the $\sigma$ bond and $\pi$ bond in graphene. (isosurface = 0.06~a.u.)}
\end{figure*}

We now consider the calculations on the periodic graphene sheet.
Figure 4 is two NLMOs (upper panel ) and OLMOs (lower panel), respectively, one for C-C $\sigma$ bond and the other for $\pi$ bond in the graphene system with 0.06 isosurface value.
The $\sigma$ bond can be well reproduced by both of the OLMO and NLMO, while both of the MOs fail to present the traditional $\pi$ bond in graphene.
While the generated NLMOs are more localized and reasonably representative of the $\pi$ bond compared with OLMOs due to its more compact property, but the shape of the MO is more like a mix between $\sigma$ and $\pi$ bond instead of a pure $\pi$ bond because of the limitation of the Boys localization functional.

It is well known that OLMOs attain orthogonality tails whose presence is crucial for OLMOs to satisfy the orthogonal condition, while NLMOs which remove the orthogonalization constrain tend to get smaller tails and more compact orbitals. 
To illustrate and compare the tail difference between OLMOs and NLMOs, we plot the NLMO and OLMO of C$_{72}$H$_{24}$ with smaller isosurface value (0.003).
According to the Fig. 5, we can conclude that the thickness of the orthogonality tails is reduced for the localized NLMOs compared to the orthogonal counterpart.

In summary, we found that NLMOs are significantly more compact and less tails than corresponding OLMOs based on our newly proposed black-box method. 
We successfully developed a new approach, which minimize the Boys localization functional and automatically self-adjust penalty function, to generate NLMOs without the need of understanding the bonding patterns in the system or any pre-fixed/defined NLMO centers.
But we also encountered several problems  in our method.
We observed the convergence rate issue in some systems due to we only use the gradient instead of Hessian as the new line search direction.
Also, as we mentioned before, the Boys localization functional tends to mix the $\sigma$ and $\pi$ bond during the minimization procedure, so the $\pi$ bond of NLMOs we obtained are slightly different from the traditional bonding pattern.
This problem is possible to be solved if we use Pipek-Mezey~\cite{pipek1989a_fast} localization scheme.
Even though obtained NLMOs with our method have less orthogonality tails compared with conventional OLMOs, we still could observe the significantly existence of tails when using small isosurface value, which means the generated NLMOs are still not nonorthogonal enough to eliminate the tails when using second order localization functional. 
We hope this issue can be solved when applying higher order localization functional.

%RZK: isosurface value -> a.u. use words, not just "isosuface="
\begin{figure}[htbp]
\includegraphics[scale=0.6]{figure_5.pdf} 
  \caption{Comparison of one of the NLMOs (top) and OLMOs (bottom) or a $\pi$ bond in the conjugated C$_{72}$H$_{24}$ polymer. The isosurface value is set at a relatively low value of 0.003~a.u. to emphasize the tails of the orbitals.)}
\end{figure}


%Present localization paths for two systems: simple and challenging. Use $c_P$ values determined by our automatic procedure. Remember that the whole point of creating this method was to make the localization black-box.

%Can we localize virtual orbitals?

%Discuss existing issues: convergence rate, sigma-pi mixing, orthogonalization tails, sparsity of the final coefficients. Mention possible future work to resolve them.

%TODO: Visualize an xyz file with localization centers to make sure NLMOs are physical. Compute the $c_P$ coefficient correctly the equation to minimize the L2 norm of the gradient.

\section{Conclusions}
In this paper, we proposed an unconstrained black-box method to localize nonorthogonal orbitals that automatically determine the NLMOs' centers during the optimization process without any pre-understanding/defined bonding patterns in the system. 
The adjustable penalty function strength allows us to manually regulate the balance between orthogonality and locality of NLMOs.
Our calculations show that the optimization is stable and fast in our most tested systems.
With the help of the penalty function, we did not observe linear dependence issue among generated NLMOs.

Our results, which agree with previous research by Yang et al~\cite{feng2004An_efficient, cui2010efficient}, show that the NLMOs are around $7\%$ and $28\%$ more compact than the corresponding OLMOs and mostly accordance with the traditional chemical bond theory.
Also, we  can conclude that these NOLMOs can be used to present the electronic structure in the completely equivalent to the representations by the OLMOs and CMs.
Moreover, we found that the orthogonal tails are significantly reduced in the NLMOs compared with the OLMOs.
The faster tail decay for nonorthogonal orbitals is due to more degree of freedom are available without using the orthogonality constrains for nonorthogonal orbitals than orthogonal orbitals.

\section{Acknowledgments} 

The research was funded by the Natural Sciences and Engineering Research Council of Canada (NSERC) through Discovery
Grants (RGPIN-2016-0505). The authors are grateful to Compute Canada and, in particular, the McGill HPC Centre for computer time.

\bibliographystyle{apsrev4-1}
\bibliography{NLMOs}

\end{document}
